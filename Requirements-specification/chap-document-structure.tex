\chapter{Document structure}
\label{chap-document-structure}

In order to encourage reuse and factoring of text elements, we think
it might be a good idea to avoid fixed structuring elements such as
parts, chapters, sections, etc.  Instead, we think that text should be
structured into \emph{chunks}.  A chunk might be a word, a phrase, a
sentence, a paragraph, a math formula, a table, etc.  Furthermore, a
single structuring chunk (called, say, \emph{sequence}) would contain
a list of inferior chunks.  There would be some restrictions on what
type of chunk can be contained in what other chunk, of course.

We think that any chunk could be \emph{labeled}.  We think a label
should be a positive integer, usually a fixnum, decided by the author
of the chunk, using the following code:

\begin{verbatim}
(logior (ash (get-universal-time) 16) (random (ash 1 16)))
\end{verbatim}

Chunks can be \emph{included} in other chunks using a special
\emph{include} chunk which takes the label of some other chunk.
The meaning of inclusion is that the included chunk should appear as
if its associated text had occurred instead.  

Chunks can also be \emph{referred to} by other chunks using a
\emph{reference} chunk.  The meaning is that the chunk being referred
to should appear as a hyperlink, a page reference, etc. in the chunk
referring to it. 

A chunk can be referred to or included in several other chunks.  This
way, different structures can be imposed on the text for different
purposes, for instance a web site or a printed document. 

To allow for parameters for chunks, we imagine a particular
\emph{binding chunk} that binds a variable%
\footnote{The exact nature of variables has not been determined yet} V
to a particular chunk C, so that references or inclusions in the
subordinated chunk D containing the variable V would resolve to C.
This mechanism would replace the use of macros for simple text
inclusion.  As an example, consider the macro named ``IfYouWantClass''
in the CLIM II specification.  It is just a piece of text with three
parameters.

Just as chunks can be labeled, it should be possible to give any chunk
a \emph{title} by wrapping it in a particular \emph{title} chunk.
This title would appear when the chunk is referred to, or when it is
the chapter or section of a printed manual.

In order to encourage \emph{factoring} of various elements and in
order to make it easier for authors to include existing elements in
new documents, \sysname{} maintains a \emph{database} of labeled
elements.  This database is a mapping from labels in the form of
\commonlisp{} strings to elements.

However, it must be possible for this database to grow incrementally.
For that reason, we allow for an ordinary string to occur, whenever a
\texttt{word} or \texttt{punctuation} chunk is required.

\Defclass {chunk}

This is the base class of all chunks.

\Definitarg {:label}

This initarg is used to supply the label for the chunk.  The value of
the argument should be a positive fixnum generated like this:

\begin{verbatim}
(logior (ash (get-universal-time) 16) (random (ash 1 16)))
\end{verbatim}

\Defgeneric {label} {(chunk)}

This generic function returns the label of the chunk as supplied to
the \texttt{:label} initarg. 

\Defclass {sequence}

This class is a subclass of the class \texttt{chunk}.

\Definitarg {:children}

This initarg is used to supply the children of the chunk.  The value
of the argument should be a list of chunks.

\Defgeneric {children} {(sequence)}

This generic function returns the list of children of the chunk, as
supplied to the \texttt{:children} initarg.

\Defclass {morpheme}

This class is the lowest level class for representing meaningful
sequences of characters.  Instances of this class are used as
constituents of instances of the \texttt{word} class.

While it may seem excessive to have a class for this level of detail,
it can be useful in many situations:

\begin{itemize}
\item It is easy to think that words in English are separated by
  spaces, this is not always the case.  Some words are made up of
  several morphemes separated by space, such as ``sponge cake''
\item In languages such as Vietnamese, morphemes that make up a word
  are always separated by spaces, and the information on which
  morphemes go together is crucial for understanding the meaning of a
  phrase.
\item One might want to search for morphemes rather than for entire
  words, to avoid having to account for verb conjugations or plural
  endings.
\end{itemize}

\Definitarg {:rendering}

This initarg determines how the morpheme is to be rendered.  The value
of the argument is a string.

\Defgeneric {rendering} {(morpheme)}

This generic function returns the rendering of the morpheme as the
string supplied to the \texttt{:rendering} initarg.
