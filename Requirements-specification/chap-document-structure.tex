\chapter{Document structure}
\label{chap-document-structure}

In order to encourage reuse and factoring of text elements, we think
it might be a good idea to avoid fixed structuring elements such as
parts, chapters, sections, etc.  Instead, we think that text should be
structured into \emph{chunks}.%
\footnote{The name \emph{chunk} is preliminary, of course.}
A chunk might be a word, a phrase, a sentence, a paragraph, a math
formula, a table, etc.  Furthermore, a single structuring chunk
(called, say, \emph{sequence}) would contain a list of inferior
chunks.  There would be some restrictions on what type of chunk can be
contained in what other chunk, of course.  

We think that any chunk could be \emph{labeled}.  A label could for
instance be a \commonlisp{} string.  Using strings rather than (say)
symbols avoids any problems with packages that may not exist, and the
system for presenting the document could show references to labels in
an obvious way.

Chunks can be \emph{included} in other chunks using a special
\emph{include} chunk which takes the label of some other chunk.
The meaning of inclusion is that the included chunk should appear as
if its associated text had occurred instead.  

Chunks can also be \emph{referred to} by other chunks using a
\emph{reference} chunk.  The meaning is that the chunk being referred
to should appear as a hyperlink, a page reference, etc. in the chunk
referring to it. 

A chunk can be referred to or included in several other chunks.  This
way, different structures can be imposed on the text for different
purposes, for instance a web site or a printed document. 

To allow for parameters for chunks, we imagine a particular
\emph{binding chunk} that binds a variable%
\footnote{The exact nature of variables has not been determined yet} V
to a particular chunk C, so that references or inclusions in the
subordinated chunk D containing the variable V would resolve to C.
This mechanism would replace the use of macros for simple text
inclusion.  As an example, consider the macro named ``IfYouWantClass''
in the CLIM II specification.  It is just a piece of text with three
parameters.

Just as chunks can be labeled, it should be possible to give any chunk
a \emph{title} by wrapping it in a particular \emph{title} chunk.
This title would appear when the chunk is referred to, or when it is
the chapter or section of a printed manual.

In order to encourage \emph{factoring} of various elements and in
order to make it easier for authors to include existing elements in
new documents, \sysname{} maintains a \emph{database} of labeled
elements.  This database is a mapping from labels in the form of
\commonlisp{} strings to elements.
