\chapter{Document structure}
\label{chap-document-structure}

In order to encourage reuse and factoring of text elements, we think
it might be a good idea to avoid fixed structuring elements such as
parts, chapters, sections, etc.  Instead, we think that text should be
structured into \emph{chunks}.%
\footnote{The name \emph{chunk} is preliminary, of course.}
A chunk might be a word, a phrase, a sentence, a paragraph, a math
formula, a table, etc.  Furthermore, a single structuring chunk
(called, say, \emph{sequence}) would contain a list of inferior
chunks.  There would be some restrictions on what type of chunk can be
contained in what other chunk, of course.  

We think that any chunk could be \emph{labeled}.  A label could for
instance be a \cl{} symbol so that we can take advantage of the
package system for labeling chunks.

Chunks can be \emph{included} in other chunks using a special
\emph{include} chunk which takes the label of some other chunk.
The meaning of inclusion is that the included chunk should appear as
if its associated text had occurred instead.  

Chunks can also be \emph{referred to} by other chunks using a
\emph{reference} chunk.  The meaning is that the chunk being referred
to should appear as a hyperlink, a page reference, etc. in the chunk
referring to it. 

A chunk can be referred to or included in several other chunks.  This
way, different structures can be imposed on the text for different
purposes, for instance a web site or a printed document. 
