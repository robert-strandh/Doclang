\chapter{Introduction}
\setcounter{page}{1}%
\pagenumbering{arabic}%
\label{chap-introduction}%

Systems for writing documents roughly fall into two categories.  In
the first category, we find so called \emph{markup} systems.  Such
systems are specified in terms of the \emph{syntax} of the input
language.  Typical systems in this category are TeX, Texinfo, etc.  In
the second category, we find systems that allow the user to produce
documents \emph{interactively}.  These systems are specified in terms
of the \emph{gestures} that the user can emit in different situations.
Typical systems in this category are Word and LibreOffice.

Systems in both these categories specify only input and output.  Both
types of system generate output in some form that the user can view
using standard tools, such as PDF or HTML.  In addition, interactive
tools typically use an external format that makes it possible to save
and restore a document across editing sessions.

Unfortunately, users have very specific taste, both when it comes to
input syntax and available gestures.  For that reason, we believe it
is worth thinking about a system for manipulating documents in a
different way.

The main observation is that systems in both categories are
\emph{closed} in that they do not allow the user to provide different
ways of manipulating the document in the form that the system uses
internally.  The reason is probably that these systems are written in
static languages, and so can not be easily customized to process a
document in a way different from what the initial author of the system
intended.

The purpose of this project is to create a library that defines a set
of classes and generic functions that allow some client code to create
and manipulate a document.  Contrary to the types of systems cited
above, the specification of this system is thus in terms of what kind
of objects it manipulates, and what functions exist to manipulate
them.

A library like this can then be used both by an application that reads
some markup syntax and produces the document in the form of a graph of
class instances, and by an interactive application that allows the
user to create the document by issuing gestures.  In fact, it will be
possible to create several different application with different
markups and with a different set of possible gestures.

\sysname{} is a system for structuring documents, and in particular
software documentation.  Here is what \sysname{} is \emph{not}:

\begin{itemize}
\item It is not a system for \emph{extracting documentation strings}
  from Common Lisp source code and turn it into HTML or some other
  format.
\item It is also not another system defined in terms of the
  \emph{input syntax} that it accepts, as is the case for existing
  markup systems.
\item Finally, it is not an interactive system defined in terms of the
  \emph{gestures} that the user is allowed to emit.
\end{itemize}

Instead, it is a system that is defined in terms of a set of \clos{}
\emph{protocols}%
\footnote{A \clos{} protocol consists of a set of \emph{protocol
    classes} and a set of \emph{generic functions} operating on
  instances of those classes.}
defined on a data structure kept in main memory.
